
\footnotesize
\begin{longtblr}[
  caption = {Codebook for Cleaned Indian Census Population Dataset},
  label = {tab:codebook},
]{
  width = \textwidth,
  leftsep = 0pt,
  rightsep = 0pt,
  colsep = 6pt,
  rowsep = 10pt,
  colspec = {
    Q[l,wd=0.11\textwidth]
    Q[l,wd=0.18\textwidth]
    Q[l,wd=0.14\textwidth]
    Q[c,wd=0.06\textwidth]
    Q[l,wd=0.10\textwidth]
    Q[c,wd=0.06\textwidth]
    Q[l,wd=0.20\textwidth]
  },
  rowhead = 1,
  hlines,
}
Variable &
Description &
Unit of Observation &
Type &
Valid Range / Categories &
Missingness &
Notes \\
ent\_id & Unique identifier for each administrative entity in the dataset & Administrative entity & String (Identifier) & See notes & 0 (0.0\textbackslash\{\}\%) & Constructed identifier representing a single administrative entity (India, state/UT, district, or subdistrict); each entity appears multiple times across rows due to different urbanisation classes. Generated by concatenation of st\_code, dist\_code, and sbd\_code; not part of original census coding. \\
unique\_id & Row-level unique identifier for each administrative entity and its urbanisation component & Administrative entity × urbanisation class & String (Identifier) & See notes & 0 (0.0\textbackslash\{\}\%) & Constructed identifier that uniquely identifies each row in the dataset. Generated by concatenation of st\_code, dist\_code, sbd\_code, and the first letter of dem\_class; not part of original census coding. \\
st\_code & Official census state or union territory code & State / Union Territory & String (Identifier) & See notes & 0 (0.0\textbackslash\{\}\%) & Highest-level administrative identifier in the census hierarchy. Encoded as a two-digit string; coded as '00' for aggregate India-level entries. Stable across all constituent districts and subdistricts. \\
dist\_code & Official census district code within each state or union territory & District & String (Identifier) & See notes & 0 (0.0\textbackslash\{\}\%) & Middle-level administrative identifier in the census hierarchy. Encoded as a three-digit string; coded as '000' for India-level and state/UT-level entries. Stable across all constituent subdistricts. \\
sbd\_code & Official census subdistrict (tehsil/taluka) code within each district & Subdistrict & String (Identifier) & See notes & 0 (0.0\textbackslash\{\}\%) & Lowest-level administrative identifier in the census dataset. Encoded as a five-digit string; coded as '00000' for India-level, state/UT-level, and district-level entries. Code '99999' denotes 'Area not under any Sub-district', used for geographic areas not formally categorized into standard subdistrict units. \\
ent\_type & Administrative level of the entity represented in the row & Administrative entity & Categorical / String & DISTRICT, INDIA, STATE, SUB-DISTRICT & 0 (0.0\textbackslash\{\}\%) & Categorical variable with values: 'INDIA', 'STATE', 'DISTRICT', and 'SUB-DISTRICT'. Indicates the hierarchical level of the administrative unit. \\
name & Official name of the administrative unit & Administrative entity & Categorical / String & See notes & 0 (0.0\textbackslash\{\}\%) & Names standardized for formatting and whitespace consistency during cleaning. \\
dem\_class & Urbanisation demographic classification of the administrative entity & Administrative entity × urbanisation class & Categorical / String & RURAL, TOTAL, URBAN & 0 (0.0\textbackslash\{\}\%) & Categorical variable with values 'TOTAL', 'RURAL', and 'URBAN'. Each administrative entity is represented by up to three rows corresponding to these classes; 'TOTAL' represents the sum of rural and urban components where applicable and defines the structural logic of the dataset. \\
ivlg\_count & Number of inhabited villages within the administrative unit & Administrative entity × urbanisation class & Integer & 0 -- 597608 & 0 (0.0\textbackslash\{\}\%) & Non-zero values occur only for 'RURAL' and 'TOTAL' entries. Always zero for 'URBAN' entries by definition. May be zero for 'RURAL' or 'TOTAL' entries if no rural component exists or for special administrative entities that function as neighbourhoods or commercial centres. \\
uvlg\_count & Number of uninhabited villages within the administrative unit & Administrative entity × urbanisation class & Integer & 0 -- 43324 & 0 (0.0\textbackslash\{\}\%) & Non-zero values occur only for 'RURAL' and 'TOTAL' entries. Always zero for 'URBAN' entries. May be zero for rural entities with no uninhabited villages or for special administrative entities functioning as neighbourhoods or commercial centres. \\
town\_count & Number of census or statutory towns within the administrative unit & Administrative entity × urbanisation class & Integer & 0 -- 7933 & 0 (0.0\textbackslash\{\}\%) & Non-zero values occur only for 'URBAN' and 'TOTAL' entries. Always zero for 'RURAL' entries by definition. May be zero for 'URBAN' or 'TOTAL' entries when no urban component exists or for special neighbourhood or commercial-centre entities. \\
hhld\_count & Total number of households within the administrative unit & Administrative entity × urbanisation class & Integer & 0 -- 249501663 & 0 (0.0\textbackslash\{\}\%) & May be zero for 'RURAL' or 'URBAN' entries when that component does not exist within an administrative entity. Always non-zero for 'TOTAL' entries, including special neighbourhood or commercial-centre entities. \\
total\_pop & Total resident population of the administrative unit & Administrative entity × urbanisation class & Integer & 0 -- 1210854977 & 0 (0.0\textbackslash\{\}\%) & May be zero for 'RURAL' or 'URBAN' entries when that component does not exist. Always non-zero for 'TOTAL' entries, indicating the absence of ghost administrative units. \\
male\_count & Total male resident population of the administrative unit & Administrative entity × urbanisation class & Integer & 0 -- 623270258 & 0 (0.0\textbackslash\{\}\%) & May be zero for 'RURAL' or 'URBAN' entries when that component does not exist. Always non-zero for 'TOTAL' entries. \\
female\_count & Total female resident population of the administrative unit & Administrative entity × urbanisation class & Integer & 0 -- 587584719 & 0 (0.0\textbackslash\{\}\%) & May be zero for 'RURAL' or 'URBAN' entries when that component does not exist. Always non-zero for 'TOTAL' entries. \\
area\_sqkm & Geographic area of the administrative unit in square kilometers & Administrative entity × urbanisation class & Float & 0.0 -- 3287469.0 & 117 (0.59\textbackslash\{\}\%) & May be missing for certain entities due to census data limitations. For some 'TOTAL' entries, rural and urban areas may not sum perfectly due to very small floating-point discrepancies. May be zero for non-existent 'RURAL' or 'URBAN' components but never zero for 'TOTAL' entries. \\
pop\_per\_sqkm\_old & Population density as reported in the raw census source & Administrative entity × urbanisation class & Float & 0.0 -- 89185.0 & 117 (0.59\textbackslash\{\}\%) & May be missing for the same entities where area data is unavailable. Retained for reference despite mixed integer and float values arising from inconsistent rounding in the raw census source. May be zero for non-existent 'RURAL' or 'URBAN' components but never zero for 'TOTAL' entries. \\
pop\_per\_sqkm\_new & Recomputed population density based on cleaned population and area values & Administrative entity × urbanisation class & Float & 0.0 -- 89185.0216047336 & 117 (0.59\textbackslash\{\}\%) & May be missing for the same entities where area data is unavailable. Computed as total population divided by area; consistently stored as float. Equals zero when area is zero due to non-existence of a component. Never zero for 'TOTAL' entries. \\
\end{longtblr}